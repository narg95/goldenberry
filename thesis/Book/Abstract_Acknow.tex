\chapter*{Acknowledgements}
\begin{abstract}
This thesis focuses on the problem of estimating relevance of observed variables for a classification task in high dimensional spaces, which is known as feature subset selection or feature relevance determination by the machine learning community. The main goal of the thesis was to design of a novel feature relevance estimation method by combining techniques for estimation of dependency networks with weighted kernel machines\footnote{\tiny The topics of feature selection methods, estimation of probability distribution algorithms and kernel machines are briefly reviewed in section \ref{sec:lecrev}. The interested reader is referred to \cite{guyon03,heckerman99, cristianini04} for details.}. The method, called \WKII, works within a population-based stochastic search framework where relevant (but unknown) variables from a given dataset\footnote{\tiny In the context of this paper \emph{dataset} means data arranged in a tabular way where columns represent variables and rows represents instances.  Other types of formats (such as text, images, graphs, etc.) are not considered.} are found by iteratively evolving a set of relevance estimation candidates. These candidates will consist of parameters of bivariate conditional probability distributions. The distributions could be seen as dependency networks of how variables influence each other.  With this information a subset of the most relevant features from the original sample can be selected to perform classification, the suitability of the subset being assessed as its predictive classification accuracy. Weighted kernel classifiers are used for this purpose. The method provides additional information about dependency among such variables as a byproduct of the selection process.  The thesis also contributes to the development of \mbox{\TILDA}, a novel estimation of distribution algorithm, and \mbox{\GB}, a supplementary machine learning and evolutionary computation suite for \Orange.
\end{abstract}
\selectlanguage{spanish}
\begin{abstract}
Aca se encontrara el resumen en español del Abstract.
\end{abstract}
\selectlanguage{english}
\pagenumbering{arabic}